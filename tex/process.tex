% Comment to allow building from within this file
%!TEX root = main.tex

\blankscreen{
  Process\textCR
  \textCR
  Much to learn -- Agile, Waterfall, Critical Path, Critical Chain, XP, Scrum, Prince2\textCR
  Take agile development -- Let's look at the fundamentals...
}

\begin{frame}
  \frametitle{Manifesto for Agile Software Development}

  {\footnotesize \color{black} We are uncovering better ways of developing
  software by doing it and helping others do it.
  Through this work we have come to value:}

  \vfill

  \tikzmarkin<2->{a1}(0.1, -0.2)(-0.1, -0.1)
  \begin{description}
    \begin{small}
      \item [Individuals and interactions] over processes and tools\tikzmarkend{a1}
      \item [Working software] over comprehensive documentation
      \item [Customer collaboration] over contract negotiation
      \item [Responding to change] over following a plan
    \end{small}
  \end{description}

  \vfill

  {\footnotesize That is, while there is value in the items on
  the right, we value the items on the left more.}

  \begin{block}{}
    {\tiny
      Kent Beck\hspace{5pt}
      Mike Beedle\hspace{5pt}
      Arie van Bennekum\hspace{5pt}
      Alistair Cockburn\hspace{5pt}
      Ward Cunningham\hspace{5pt}
      Martin Fowler\hspace{5pt}
      James Grenning\hspace{5pt}
      Jim Highsmith\hspace{5pt}
      Andrew Hunt\hspace{5pt}
      Ron Jeffries\hspace{5pt}
      Jon Kern\hspace{5pt}
      Brian Marick\hspace{5pt}
      Robert C. Martin\hspace{5pt}
      Steve Mellor\hspace{5pt}
      Ken Schwaber\hspace{5pt}
      Jeff Sutherland\hspace{5pt}
      Dave Thomas\\
      © 2001, the above authors
      this declaration may be freely copied in any form, but only in its entirety through this notice.}
  \end{block}

  \speakernote{
    The original agile manifesto\textCR
    Look what they chose as their very first item\textCR
    ** Don't be a Git **
  }
\end{frame}

\againframe<2-3>{style_plot}

